% This file was generated automatically by pandoc. Do not edit manually.
%-----------------------------------------------------------------------

DougDoug is not known as a poet, even among his fans. Yet this is
precisely what makes him such an invaluable candidate for authentic
analysis. As G. K. Chesterton observed, the democratic
impulse\index{democratic impulse} must be felt not just in man's most
important affairs, but also in his least important
\autocite[73]{chesterton_orthodoxy}. If he cannot blow his own nose,
dance his own jig, or write his own love letters, what hope does he have
of running his own government? The populist-romantic impulse must
ultimately be embodied or it loses its essential quality. In the age of
spectacle this real, lived experience of poetry has gone
under-appreciated. Even more, as Pope affirms, ``there is hardly any
human creature past childhood, but at one time or other has had some
poetical evacuation, and, no question, was much the better for it in his
health'' \autocite[8]{pope_peri_bathous}.

Yet we must still ask, why all this fuss about counting syllables and
splitting sentences across multiple lines? There are many reasons, but
the most important is of course to take up more pages with only a small
amount of story so that the bookseller can charge more for the same
amount of words. Unfortunately, the marketplace has made this natural
and necessary practice seem an artificial and dull inefficiency, so that
Serious Literature must now be packed so finely onto a page that no one
bothers with actually reading it. True poets prefer to give their words
the room to roam around as they please.

The next most important reason is that there are things that you just
cannot say except in a poem. Such a line as ``My love shall in my verse
ever live young'' \autocite[Sonnet 19.14]{shakespeare}. In a novel you
might only include it as something spoken by a foolish and naïve
character, so as to make fun of Romantic sentimentalism. Prose by design
demands a diminished sentiment, as it must be an easy thing that could
have just as likely occurred as a conversation over cups of coffee. I am
reliably told that all Italian novels now are set in a café to match the
environment that they are most likely to be read in. The need to make
everything with Upmost Seriousness has made it impossible to talk of
anything really important. True emotions do not flourish in the binds of
parliamentary decorum.

But what makes \emph{poetry} special? Why can you say such a thing as
``My love shall in my verse ever live young?'' Or rather, why can't you
just say the same thing in the \emph{normal} way? ``My love lives
forever in my poetry'' seems like a fine and sturdy sentence, after all.
There even seems to be a novel inside of it, if you stare long enough.
You can see two lovers exchanging letters across time (it must be time
travel, since simple parental disapproval isn't taken seriously in our
age as the cause of a star-crossed relationship). But there's the rub.
The novel was within the line. The novel cannot contain the poetry
because the poem already contained the novel. Let us then pry open an
exemplary little chestnut to see what sort of sprawling meaning it might
contain.

\index{Death Roads|(} In his debut song ``Death Roads,'' DougDoug
delivers a scathing critique of the human condition and the tragicomic
nature of art as a means of addressing such issues. Consider just the
first four lines:

\begin{quote}
West Virginia\index{West Virginia}\\
Take me roads\\
Country roads\\
Take me death \autocite[0:09]{dougdoug_death_roads}
\end{quote}

These echo the same iconic themes found in John
\cite{denver_country_roads}'s ``Take Me Home, Country Roads,'' but
subtly subverted. Rather than the nostalgic and elegiac, we get
something anthemic and thanatic\index{death-drive}. The peaceful
reprisal to uncomplicated country life has been replaced with the
unapologetic desire for death itself.

The line ``Take me roads'' melds a complex set of emotions. Principally
it represents the abandonment of agency, which Denver's original also
delivers. But Doug introduces a grammatical ambiguity that ensures the
line cannot be fully resolved. Is he addressing the roads themselves? Is
he asking to be brought to these roads (as in the original ``Take me
home'')? Is the ``taking'' a mere physical conveyance or a more
intimate, sexual image? These unresolved questions destabilize the text,
rendering it immediately as an uncooperative object, denying the
listener any easy interpretation. That destabilization is immediately
enhanced and implicitly parodied by the next four lines:

\begin{quote}
Oh, wait\\
De, lia\\
Death Road\\
Feh sigh ding \autocite[0:18]{dougdoug_death_roads}
\end{quote}

By saying ``Oh, wait,'' the song communicates that even its author has
lost track of its structure. What follows are the phonetic suggestions
of a song. Meaning is unimportant as the lyrics dissolve semantically
while simultaneously highlighting the central concept, which appears
with striking clarity: ``Death Road.'' This enigmatic figure haunts the
text, contrasting with and usurping the place of the ``country roads.''
The variegated, classically pastoral setting cannot withstand the
monolithic, Modernist reaper-consumer\index{Modernist reaper-consumer}.
Doug reveals that Denver's roads were already ``death roads,'' deficient
from their outset. Doug continues this revelation though the
reintroduction of ``home'' in the following lines:

\begin{quote}
Oh, the Death Road\\
Take me home\\
To Death Road\\
To a road\\
Where I belong \autocite[0:27]{dougdoug_death_roads}
\end{quote}

Death Road must ``Take me home / to Death Road.'' This recursive
structure\index{recursion} links the beginning of the journey with its
end. There is no way to mark progress because the journey itself is the
destination. Death Road is ``a road / Where I belong,'' further
strengthening the song's underlying death-drive\index{death-drive}. The
road acquires an inescapable, inevitable presence. The song's semantic
instability\index{semantic instability} reaches its apogee in the next
section:

\begin{quote}
Raz bin road\\
Death Road\\
Ehhh road\\
Take me home\\
Death row \autocite[0:39]{dougdoug_death_roads}
\end{quote}

``Country roads'' have been dissolved into ``Raz bin road'' while
``Death Road'' becomes ``Ehhh road.'' Both verge into the unutterable.
``Raz bin'' itself evokes ``has been,'' perhaps signaling a lingering
sense of nostalgia leaking through from the original or the satirical
mockery of that same nostalgia. The most radical element of this section
subsists in the introduction of ``Death row.'' By invoking the image of
prisoners awaiting their execution, Doug fully strips the song of its
romanticism. We are no longer taking a gentle journey. We are locked
into a prison cell, awaiting a fate we have no hope of deferring.
Transformation has become petrification, motion has turned to stasis.

This somber image is suddenly interrupted by the single line ``Hullo?''
\autocite[0:55]{dougdoug_death_roads} This creates a textual
rupture\index{textual rupture}, a break in the performance. We have been
thrust from the bardic realm into the ``real world'' (which is itself
still an experience mediated by Internet protocols and streaming
platforms). Doug-as-Streamer replaces Doug-as-Poet, though we know that
they are the same person. By ``switching costumes'' during the
performance and presenting this interrogative, the poem draws attention
to the work \emph{as performance}, asking the audience to question
whether poetry is constructed through the right words, the right
context, or something else entirely.

In the following lines, there are two important points. The more
substantial resides in the lines: ``Take me home / To the road / I
death'' \autocite[1:13]{dougdoug_death_roads}. Here even the act to
``belong'' has been subsumed by death. Nor is the singer allowed the
agency to ``die,'' but rather the noun takes the place of the verb,
further cementing the singer's sense of helplessness. Immediately
following are the lines, ``West Death Road / Take me home / Country
row'' \autocite[1:18]{dougdoug_death_roads}. The ``West Death Road,''
which had appeared earlier, displaces the entire state of ``West
Virginia,''\index{West Virginia} making Death Road a truly omnipresent
entity.

Following this, we get these lines:

\begin{quote}
Death Road\\
Death Road\\
Death Road\\
Death\\
Death Ruh \autocite[1:29]{dougdoug_death_roads}
\end{quote}

The dreadful repetition builds tension and anticipation, while the
sudden stuttering at ``Death Ruh'' symbolizes exhaustion and confusion.
The singer is running through a maze of ``death roads,'' able only to
feebly utter: ``Take me home / To a place''
\autocite[1:41]{dougdoug_death_roads}. However, it's at this point that
the singer flips from pursued to pursuer:

\begin{quote}
Saw em\\
Death Road\\
Death Road\\
Death Road\\
To a death\\
Where I road \autocite[1:53]{dougdoug_death_roads}
\end{quote}

The interjection of ``Saw em,'' especially in the performance where it
is spoken rather than sung, creates another
rupture\index{textual rupture} to signal the reversal. ``To a death /
Where I road'' adds a double-entendre, as if the singer is now riding
toward death. No longer is the singer passively asking for death. He is
actively pursuing it. He has seen the way and takes action.

The piece's most radical move appears near its end, starting with the
lines, ``Op, there's a pig\index{pigs} / Shit''
\autocite[2:10]{dougdoug_death_roads}. Here the context of the
performance reasserts dominance. The singer is literally singing this
song while being attacked by pigs, creating an intense comedic
dissonance between the subject matter (the desire for death) and the
situation where it is performed. Yet that dissonance exposes something
critical about us as readers. We assume that to be mauled by pigs while
singing a nonsensical song is a comedic contextualization, rather than
respecting it as a real situation which could very well happen to
someone. This notion that there are ``comedic ways to die'' represents
the lingering pagan heroic mindset\index{pagan heroic mindset} within
the modern mind. Even if the idea that dying heroically in battle with
an appropriate foe as the only way into Valhalla has been roundly and
explicitly rejected, we can't help but be haunted by its instinctual
logic.

The poetical convention is further undermined with the next lines, which
have largely abandoned the pretense of song:

\begin{quote}
Ah, there's multiple of them\\
I paralyzed one\\
Paralyzing the other\\
Shit, shoot it, shoot it, shoot it, shoot it\\
For the love of God, shoot it\\
Fucking run\\
West Virginia\index{West Virginia}\\
Wait, is it dead? \autocite[2:16]{dougdoug_death_roads}
\end{quote}

The only elements that inform us that this is a song are the repetition
of ``West Virginia'' from the first line and the line breaks. The
passage asks whether these minimal formal elements alone are enough to
be ``a song.'' It evokes previous Internet traditions of remixing news
clips into music, turning the prosaic into something poetic through
recontextualizing it. As for the core meaning of this segment, it
represents the displacement of the death-drive\index{death-drive} from
the singer onto the pigs\index{pigs}. The fear and paralysis that the
speaker felt previously is enacted as deliberate violence onto them.

There's palpable fear in the lines ``Shit, shoot it, shoot it, shoot it,
shoot it / For the love of God, shoot it.'' While the song has largely
dispensed with the formal elements of lyric writing, such as meter and
rhyme, this line demonstrates an even further abandonment, leaving
behind even the song's own established conventions for a raw and
unpolished scrawl. It's no longer a polite parody but an
expletive-ridden screed that only vaguely gestures toward Denver's
original. The question ``Is it dead?'' represents here a meta-question:
Has the song committed its own deconstruction? Has the text's inherent
instability\index{semantic instability} made its own meaning
unintelligible?

The singer attempts to return to form, but the lines retain the aporetic
residue\index{aporetic residue} from the previous encounter:

\begin{quote}
Country roads\\
Ehhh, to a place\\
Where I Death Road\\
West Virginia\index{West Virginia}\\
Virgin, AHHHH \autocite[2:30]{dougdoug_death_roads}
\end{quote}

Here we recognize the interjections of ``Ehhh'' and ``AHHHH'' not as
mere lapses in memory but active interruption from the environment.
Because the singer has rejected his previous acceptance of death and
displaced his desire onto the environment, it has become actively
hostile and disruptive. ``Death Road'' is no longer a vehicle or
destination but has become the action of the singer himself. The
specific corruption of ``Virginia'' into ``Virgin'' further expresses
the singer's disgust for the unmolested, uncontrolled environment. Where
in Denver's song the country's lack of urbanization is its primary
allure, Doug inverts this value into an insult.

The song winds down with a bit of wordplay:

\begin{quote}
To the road\\
Where I road\\
To a place\\
Where the road? \autocite[2:46]{dougdoug_death_roads}
\end{quote}

These wordplays hide an important development in the song. As ``Where I
road {[}or rode{]}'' becomes ``Where the road?'' we realize that the
singer's previous confidence and determination have vanished, leaving
him in a new suspended state of placelessness. Previously he knew the
road, he knew the home, he knew ``a place / Where I belong.'' Now he
cannot even find ``the road / Where I {[}rode{]} / To a place.'' This
displacement and disorientation is absolute and complete.

The song ends with rage. In the final simmering lines we hear, ``Take me
home / To a road / Get fucked'' \autocite[2:57]{dougdoug_death_roads}.
The song is no longer evocative but desperate. Rather than a traveller
trying to get home after a long trip, it has become the desperate plea
of someone lost in the woods or stuck on a sinking ship. He wants to be
brought home, or just to a road. Failing these, he has nothing left to
say except to curse at the world in his final moments.

DougDoug's ``Death Roads'' doesn't yield to critical insights easily,
but as we can see, through careful close reading we can uncover the
invaluable themes and subjects it addresses. Hopefully it has also
inspired you to take up a hum or two for yourself, whether you do it
while showering or during a swine attack. It's the democratic thing to
do. \index{Death Roads|)}
