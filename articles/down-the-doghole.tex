DougDoug is not known as a poet, even among his fans. Yet this is
precisely what makes him such an invaluable candidate for authentic
analysis. As G. K. Chesterton observed, the democratic impulse must be
felt not just in man's most important affairs, but also in his least
important \autocite[73]{chesterton_orthodoxy}. If he cannot blow his own
nose, dance his own jig, or write his own love letters, what hope does
he have of running his own government? The populist-romantic impulse
must ultimately be embodied or it loses its essential quality. In the
age of spectacle this real, lived experience of poetry has gone
under-appreciated. Even more, as Pope affirms, ``there is hardly any
human creature past childhood, but at one time or other has had some
poetical evacuation, and, no question, was much the better for it in his
health'' \autocite[8]{pope_peri_bathous}.

In his debut song ``Death Roads,'' DougDoug delivers a scathing critique
of the human condition and the tragicomic nature of art as a means of
addressing such issues. Consider just the first four lines: ``West
Virginia / Take me roads / Country roads / Take me death''
\autocite[0:09]{dougdoug_death_roads}. These echo the same iconic themes
found in John Denver's ``Take Me Home, Country Roads,'' but subtly
subverted. Rather than the nostalgic and elegiac, we get something
anthemic and thanatic. The peaceful reprisal to uncomplicated country
life has been replaced with the unapologetic desire for death itself.

\subsubsection{In Defense of Poetry}\label{in-defense-of-poetry}

Why not a novel? Why all this fuss about counting syllables and
splitting sentences across multiple lines? There are several reasons,
but the first is of course to take up more pages with only a small
amount of story so that the bookseller can charge more for the same
amount of words. Unfortunately, the marketplace has made this natural
and necessary practice seem an artificial and dull inefficiency, so that
Serious Literature must now be packed so finely onto a page that no one
bothers with actually reading it. I prefer to give my words the room to
roam around as they please.

The next most important reason is that there are things that you just
cannot say in a novel and only occasionally in a play. Such a line as
``My love shall in my verse ever live young.'' In a novel you might only
include it as something spoken by a foolish and naive character, so as
to make fun of Romantic sentimentalism. Prose by design demands a
diminished sentiment, as it must be an easy thing that could have just
as likely occurred as a conversation over cups of coffee. I am reliably
told that all Italian novels now are set in a café to match the
environment that they are most likely to be read in. The need to make
everything with Upmost Seriousness has made it impossible to talk of
anything really important. All good art is based on a broken heart, and
true emotion does not flourish in the binds of parliamentary decorum.

But what makes \emph{poetry} special? Why can you say such a thing as
``My love shall in my verse ever live young''? Or rather, why can't you
just say the same thing in the \emph{normal} way? ``My love lives
forever in my poetry'' seems like a fine and sturdy sentence, after all.
There even seems to be a novel inside of it, if you stare long enough.
You can see two lovers exchanging letters across time (it must be time
travel, since simple parental disapproval isn't taken seriously in our
age as the cause of a star-crossed relationship). But there's the rub.
The novel was within the line. The novel cannot contain the poetry
because the poem already contained the novel.

Having established the indisputable superiority of poetry, it's with
much greater trepidation that I approach the question of ``What
\emph{is} poetry?'' The simple answer is that it is a magic spell. Or,
if you like, a pious prayer. It's natural for there to be certain rules
for addressing such strange entities as gods, demons, or women. Poetry
thus had a very practical purpose, as without it there might be no crop
harvest or childbirth. In the present age that purpose has been replaced
by fertilizer and fertility drugs which are much simpler to use but I
suspect much less effective overall. With its decline in use, the Rules
of Poetry have also faded from memory.

The present pottage of poets thus can be grouped into two loose camps:
The first took the opportunity to devise, in secret, their own arcane
and complex set of rules by which to write. These authors rebuff any
attempt to uncover the shape of their tools by claiming that there are
no rules. This of course makes it a most impressive feat that their
poetry---among such a diverse array of authors---all congeals into the
same-tasting substance.

The second group of poets says, ``Oh, but there were rules.'' When
asked, ``Ah, but what were they?'' they think long and hard and consult
a long list of illustrious authors who died before them. Once satisfied
with what they've found, they will say: ``Oh, you never rhyme a
preposition with an adjective'' or some other small glimmer of wisdom.
At least until another replies: ``Ah, but Shakespeare did it.'' At which
point either they must say: ``Oh, but that was Shakespeare'' or ``Oh, so
it wasn't so.'' Really they are the same response. You can think of it
as a young girl who looks over from her little table at her mother's tea
party and, arranging her toys as best she can, says: ``Oh, so mommy
pours it like so.''

I use this analogy as an apology for my own practice. Rather than sit
down with the little girl and follow her lead (and she has been
practicing for a very long time, and thus is very good at putting on her
little tea parties!), I prefer to play the secret game by myself and
make a private tea party for myself, always questioning some sort of
practice (are stuffed animals truly necessary?). This I hope has lead to
something truly unique, even if they aren't the most pleasant tea
parties.

Some critics may call these ``achey allegories,'' ``rhetorical
sleight-of-hand,'' and ``straw men.'' I can only reply that we live in a
more civilized age where the practice of burning your interlocutors at
the stake has fallen out of favor. Providing a straw effigy in their
place best balances the need for violent rebuke and the unromantic
desire of our contemporary poets to avoid pain, death, and public
accountability. There are a few Andrew Carnegies who have proposed
``steel men'' for this purpose, owing to their durability and
reusability. Undoubtedly, this ecological concern is a flimsy mask for
their industrialist intensions. I for one treasure the traditional
approach. Much like my Christmas trees, I prefer that my opponents
retain their unpredictable combustibility.

Given this admission, I do still have plans to counteract any
unpleasantness that a sensitive reader may experience. The action is
suffused with contradictory emotions, so that the hottest passions are
tempered by the coldest ironies, providing the reader with a perfectly
tepid bathos. This ensures maximal comfort while the reader's mind
drowns in its eisegetic search for subtext. Furthermore, I have employed
words which have forgotten meanings or which have more than one meaning,
so that if any part of the story should offend the readers, they can
substitute an alternative meaning to make the story more agreeable to
them. With a small amount of effort this can be applied to every word,
thus making the work a most marvelous story within the reader's mind.
Having dissolved the text and no longer in need of the book, the reader
can employ its paper to better uses, such as wrapping gifts, packing
glassware, or kindling for his own poet-burning parties.
