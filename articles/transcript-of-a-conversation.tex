% This file was generated automatically by pandoc. Do not edit manually.
%-----------------------------------------------------------------------

The subsequent article was received as a cassette tape with the above
title attached. The editors have transcribed it to the best of their
abilities. The first voice appears to be Mr.~Gregory, while the other
voices remain unidentified. They have been labelled for the reader's
convenience.

\begin{center}\rule{0.5\linewidth}{0.5pt}\end{center}

\textbf{Lucien}: {[}clears throat{]} Ah, crap I gotta get this in by
tomorrow. What did I have in mind? Oh, right. As I've argued before:

\begin{quote}
An artist is identical with an anarchist.\footnote{The editors note that
  this monologue appeared previously in Chesterton's \emph{The Man Who
  Was Thursday} and have cited it as such, despite Mr.~Gregory's
  protestations.} You might transpose the words anywhere. An anarchist
is an artist. The man who throws a bomb is an artist, because he prefers
a great moment to everything. He sees how much more valuable is one
burst of blazing light, one peal of perfect thunder, than the mere
common bodies of a few shapeless policemen. An artist disregards all
governments, abolishes all conventions. The poet delights in disorder
only. If it were not so, the most poetical thing in the world would be
the Underground Railway. Why do all the clerks and navvies in the
railway trains look so sad and tired, so very sad and tired? I will tell
you. It is because they know that the train is going right. It is
because they know that whatever place they have taken a ticket for, that
place they will reach. It is because after they have passed Sloane
Square they know that the next station must be Victoria, and nothing but
Victoria. Oh, their wild rapture! oh, their eyes like stars and their
souls--- \autocite[15-16]{chesterton_thursday}
\end{quote}

\textbf{Slavonj}: Disgusting!

\textbf{Lucien}: {[}continuing{]} Their souls again in Eden\ldots{}

\textbf{Slavonj}: Filthy, disgusting!

\textbf{Lucien}: Filthy? I'll have you know it is a very well-laundered
speech. Hm\ldots{} Of course! Now I have lost my train of
thought\ldots{}

\textbf{Slavonj}: Here, perhaps\ldots{} {[}the sound of metal clanking
and rustling paper{]} Indeed--- {[}sniffles{]} \emph{This} is true
ideology!

\textbf{Kerm}: We can't eat \emph{that}.

\textbf{Slavonj}: What is a trash can? It's an object shaped by
\emph{ideology}. What is it for? It is the situation-product, a
receptacle for society's detritus. We fill it in with this so-called
``trash'' and so it acquires such a reputation. But what if we empty the
object of its ideology? We empty the trash can and place in it something
valuable. Does it become trash? It depends on how willing you are to dig
through this trash, to tolerate its smell, and so on\ldots{}

\textbf{Kerm}: Can you, though? Y'know, there's \emph{rules}\ldots{}

\textbf{Slavonj}: So you say. Because, because--- It's a hideous thing,
I don't even want to talk about it. But we are really haunted by this
specter, this image of a ``trash man'' who will come and steal from us
this ``trash'' which we have just newly found to be valuable. This is
the violence of the oppressive classes.

\textbf{Kerm}: Will you \emph{stop} throwing things on the ground? This
place is already filthy. Please, just help me clean this up. A
disordered environment leads to a disordered mind. \emph{I} can hardly
think after looking around. The tragedy of the commons on full display.
This is exactly why we need hierarchy and not any of that Marxist
\emph{nonsense}. The biological-anthropological data are clear, they're
all around us! Indeed, you make a clean division between an exploitative
``trash man'' and an exploited ``trash recoverer,'' but the situation is
not so clear-cut. Perhaps the trash man is being exploited too by having
to bare all of society's burdensome trash. Trash is indeed a very
natural concept. Even fish and birds can identify and remove trash from
their environment.

\textbf{Slavonj}: Mm, yes. The bird clears out his spot, plucks out all
his unwanted sticks, and then he makes his nest out of my discarded
condom wrapper, no? Why not? He will pluck the colorful papers and
strings we throw away to impress upon the female birds his sexual
potency, his vital energy.

\textbf{Kerm}: {[}pause{]} What's that?

\textbf{Slavonj}: Oh, I don't know. It appears to be a brick-shaped mug.
Precisely. This is precisely my point. Do you see it? There is no
drinking hole. It is a brick that cannot be used as a brick and a mug
that cannot be used as a mug. Pure excess without obligation.
{[}sniff{]}

\textbf{Kerm}: Excessive, yes. And \emph{useless}. Who would make such a
thing? This is exactly the kind of product a proper business manager
would avoid making. The other day I had seen---see, it was on a TV in a
store window---but it was the most absurd thing. It was just a
live-streamed video of a chair in a hot tub. That was it! But there they
were, tens of thousands of people watching this chair just\ldots{}
\emph{sit there}!

\textbf{Slavonj}: This is because it's his \emph{seat}, his chair, that
gives the streamer authority. He's given that authority from the
viewers, you see. The seat already had all the power, all the potential,
and it was the streamer who used that power and authority to project his
persona. {[}snort{]} The chair stream asks the question, ``What if we do
not need the streamer at all?'' It's the true proletariat uprising, the
real worker-consumers rising to their proper place. What is a show
without its audience? Nothing. So indeed, the logical conclusion is that
we turn off the television and gaze into the mirror instead! Then we
shall see the \emph{objet petit a}\ldots{}
