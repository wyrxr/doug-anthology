% This file was generated automatically by pandoc. Do not edit manually.
%-----------------------------------------------------------------------

The subsequent article was received as a cassette tape with the above
title attached. The editors have transcribed it to the best of their
abilities. The first voice appears to be Mr.~Gregory, while the other
voices remain unidentified. They have been labelled for the reader's
convenience.

\begin{center}\rule{0.5\linewidth}{0.5pt}\end{center}

\textbf{Lucian:} {[}clears throat{]} Ah, crap I gotta get this in by
tomorrow. What did I have in mind? Oh, right. As I've argued before:

\begin{quote}
An artist is identical with an anarchist\index{anarchist}.\footnote{The
  editors note that this monologue appeared previously in G. K.
  Chesterton's \emph{The Man Who Was Thursday} and have cited it as
  such, despite Mr.~Gregory's protestations.} You might transpose the
words anywhere. An anarchist is an artist. The man who throws a bomb is
an artist, because he prefers a great moment to everything. He sees how
much more valuable is one burst of blazing light, one peal of perfect
thunder, than the mere common bodies of a few shapeless policemen. An
artist disregards all governments, abolishes all conventions. The poet
delights in disorder only. If it were not so, the most poetical thing in
the world would be the Underground Railway. Why do all the clerks and
navvies in the railway trains look so sad and tired, so very sad and
tired? I will tell you. It is because they know that the train is going
right. It is because they know that whatever place they have taken a
ticket for, that place they will reach. It is because after they have
passed Sloane Square they know that the next station must be Victoria,
and nothing but Victoria. Oh, their wild rapture! oh, their eyes like
stars and their souls--- \autocite[15-6]{chesterton_thursday}
\end{quote}

\textbf{Slavonj:} Disgusting!

\textbf{Lucian:} {[}continuing{]} Their souls again in Eden\ldots{}

\textbf{Slavonj:} Filthy, disgusting!

\textbf{Lucian:} Filthy? I'll have you know it is a very well-laundered
speech. Hm\ldots{} Of course! Now I have lost my train of
thought\ldots{}

\textbf{Slavonj:} Here, perhaps\ldots{} {[}the sound of metal clanking
and rustling paper{]} Indeed--- {[}sniffles{]} \emph{This} is true
ideology!

\textbf{Kerm:} We can't eat \emph{that}.

\textbf{Slavonj:} What is a trash can? It's an object shaped by
\emph{ideology}. What is it for? It is the situation-product, a
receptacle for society's detritus. We fill it in with this so-called
``trash'' and so it acquires such a reputation. But what if we empty the
object of its ideology? We empty the trash can and place in it something
valuable. Does it become trash? It depends on how willing you are to dig
through this trash, to tolerate its smell, and so on\ldots{}

\textbf{Kerm:} Can you, though? Y'know, there's \emph{rules}\ldots{}

\textbf{Slavonj:} So you say. Because, because--- It's a hideous thing,
I don't even want to talk about it. But we are really haunted by this
specter, this image of a ``trash man'' who will come and steal from us
this ``trash'' which we have just newly found to be valuable. This is
the violence of the oppressive classes.

\textbf{Kerm:} Will you \emph{stop} throwing things on the ground? This
place is already filthy. Please, just help me clean this up. A
disordered environment leads to a disordered mind. \emph{I} can hardly
think after looking around. The tragedy of the commons on full display.
This is exactly why we need hierarchy and not any of that Marxist
\emph{nonsense}. The biological-anthropological data are clear, they're
all around us! Indeed, you make a clean division between an exploitative
``trash man'' and an exploited ``trash recoverer,'' but the situation is
not so clear-cut. Perhaps the trash man is being exploited too by having
to bare all of society's burdensome trash. Trash is indeed a very
natural concept. Even fish and birds can identify and remove trash from
their environment.

\textbf{Slavonj:} Mm, yes. The bird clears out his spot, plucks out all
his unwanted sticks, and then he makes his nest out of my discarded
condom wrapper, no? Why not? He will pluck the colorful papers and
strings we throw away to impress upon the female birds his sexual
potency, his vital energy.

\textbf{Kerm:} {[}pause{]} What's that?

\textbf{Slavonj:} Oh, I don't know. It appears to be a brick-shaped
mug\index{BrickBrick MugMug}. Precisely. This is precisely my point. Do
you see it? There is no drinking hole. It is a brick that cannot be used
as a brick and a mug that cannot be used as a mug. It's a true
philosophical enigma. An answer in search of a question. Pure excess
without obligation. {[}sniff{]}

\textbf{Kerm:} Excessive, yes. And \emph{useless}. Who would make such a
thing? This is exactly the kind of product a proper business manager
would avoid making.

{[}A barking dog and other pedestrians obscures the conversation for the
next five minutes.{]}

\textbf{Kerm:} The other day I had seen---see, it was on a TV in a store
window---but it was the most incredible thing. It was just a
live-streamed video of a chair in a hot tub.\index{chair stream} That
was it! But there they were, tens of thousands of people watching this
chair just\ldots{} \emph{sit there}! And they seemed happier to watch
the empty chair than the person actually sitting in it.

\textbf{Slavonj:} This is because it's his \emph{seat}, his chair, that
gives the streamer authority. He's given that authority from the
viewers, you see. The seat already had all the power, all the potential,
and it was the streamer who used that power and authority to project his
persona. {[}snort{]} The chair stream asks the question, ``What if we do
not need the streamer at all?'' It's the true proletariat uprising, the
real worker-consumers\index{worker-consumer} rising to their proper
place. What is a show without its audience? Nothing. So indeed, the
logical conclusion is that we turn off the television and gaze into the
mirror instead! Then we shall recognize our \emph{Ideal ego}\ldots{}
\index{Ideal ego}

Of course, this never happens in reality. Why? Well, we must consider
the ideology of the stream. We may be tempted to analyze this and assume
that the viewers, this audience, engages with this stream to see this
Big Other\index{Big Other} that is the streamer. But this would be a
mistake. The Big Other is not the streamer. It is the viewers, the
``Chat,''\index{Chat} if you will, themselves. But it is not those
viewers as actually existing individuals. It is an idealized,
non-existent Chat that can deliver a coherent action plan even when
really-existing, material viewers are in active rebellion against the
streamer. So you see when chat goes on ``strike''\index{strike} and
fills up their box with these ellipses, the streamer can freely
``ban''\index{ban} them without ideological inconsistency. The streamer
creates a performance for the idealized viewer and purposefully removes
the real viewers when they do not match that idealized vision.

The viewers, of course, take a perverse pleasure in this ``banning,''
even when it is meted out as an explicit punishment. The system
represents a complete object, and their participation necessitates the
possibility---no, the inevitability---of the banning. This martyrdom
creates the romance of the stream. Otherwise, if they continue on, they
may watch a while, then they get bored, then they move on, and so on.
But if they are banned, then the tragedy of their divorce lifts up this
streamer-viewer couple into true tragedy, into an immortal relationship
which can confirm and is confirmed by the stream's ideology.

The chair is precisely the guarantee of the Big Other. The message of
the stream is here radically atheist. It's the death of the streamer.
It's not any kind of redemption or commercial affair, in the sense that
Doug suffers to pay for our bits. Pay to whom? For what? and so
on\ldots{} It's simply the disintegration of the streamer which
guarantees the meaning of our lives, and that's the meaning of that
famous phrase ``we solve problems that no-one has.'' In other words, the
perfect encapsulation of an enigma, the answer forever in search of the
question it answers. This is true philosophy.

\textbf{Kerm:} Really, it's a wonderful microcosm for how psychological
development operates in the real world. The streamer is the
father-figure, the structural hierarchy that defines our perspective.
The so-called ``chair stream''\index{chair stream} is a classic case of
Freudian patricide. In the quest to become truly human, we have this
shadow looming over us, this fatherly presence that subsumes and
suppresses our own shadow-selves.\index{shadow-self} By breaking the
hold over our development that the streamer-father has over us, we are
finally able to integrate our shadow-self and live an authentic---

{[}The sound of traffic obscures the next ten minutes of audio.{]}

\textbf{Kerm:} ---has become the one in need of rescue. So what does it
mean to rescue him? By abandoning the stream, the streamer has imposed
suffering upon the chat. The chair is sort of the symbol of the place of
maximal suffering. Okay? So you accept that as a challenge not as
something that you're victimized by. Maybe you accept that as the price
of existing in the world. The two major problems that people face in the
world are suffering (tragedy) and malevolence. Now malevolence comes at
a price. Truly coming in contact with evil changes you, even damages
you. But if you can face that evil, if you can face that suffering, that
opens the door to your maximal potential. Although the suffering is
great and the malevolence is deep, your capacity to transcend it is
stronger.

When the streamer ultimately returns, he becomes the embodiment of all
that malevolence. The viewer must impute these ultimately evil
attributes---drug use, robbery, murder, war crimes---so that they can
confront them in a controlled way. This is the real definition of
``entering the belly of the beast.'' A strictly materialist viewpoint
would suggest that there's no such thing as ``Doug,'' but this overlooks
the overarching symbolic importance of the streamer-archetype.
``Chat,''\index{Chat} this protagonistic role that embodies all of the
hopes and dreams of the viewers, must be pitted against an antagonistic
``Doug'' that represents all of the obstacles or problems the viewers
themselves must face in their own lives. Like a dream this Doug is
presented over and over again until the individual viewer is able to
resolve the issue.

What's striking is how this revelation is delivered through the mode of
absence. I'd consider this a particularly postmodern,
neo-Marxist\index{Postmodern Neo-Marxist} turn, like the knight who
ascends the tower to fight the wizard but finds it empty. Or he goes up
expecting a princess who turns only to find out she has been studying
all the wizard's ancient texts while she was locked up in his tower and
now has become an evil sorceress herself, which is of course framed as a
good and empowering thing. The knight still needs to fulfill his quest.
He's still trying to confront his problems, but he has been deprived of
the symbols necessary to do so. The viewer still must confront and
integrate their shadow-self,\index{shadow-self} but the stream makes
this impossible by imposing these arbitrary, nonsensical rules and then
banning anyone who does not follow them. The result is a weak-willed,
collectivized mass that substitutes its own individuation and agency for
the collective fantasy.

{[}The sound of a bus breaking and its door opening.{]}

\textbf{Slavonj:} And I agree with you, but I have to ask: What is this
that you call the ``Postmodern
Neo-Marxist?''\index{Postmodern Neo-Marxist} Because it is precisely the
turn you suggest that Marxism\index{Marxism} rejects. The knight---this
exemplary figure of male chauvinism\index{male chauvinism}---goes up and
realizes he is the villain of the story. Or at least he finds out that
there is no antagonistic force that he can define his own heroic
qualities in contradiction to. All well and good, perhaps. Where the
complexity lies is when this story displaces the critique of the feudal
or capitalist system itself. The princess can remain a princess, with
all the structure and servitude it entails, as long as she is also a
``girl boss,'' a projection of the omnipotent mother. So you see the
story is not Marxist\index{Marxism} at all. By substituting the real
economic interests with these ephemeral identity interests, capitalism
can continue having its crises and further consuming the consumer so
long as it can construct these identity costumes and redirecting
material economic conflicts onto them. Marx\index{Marx} did not have
such a simplistic view of equality.

{[}The sound of a bus driving away.{]}

\textbf{Slavonj:} We must go back to your point that the viewers seemed
happier with the empty chair. I think that our first task should be to
problematize the notion of happiness itself. Happiness\index{happiness}
is a byproduct of our true goals. It can never be a pure goal in itself.
Once we accept a pursuit of happiness divorced from the context of the
Real, we've accepted a fantasy.

\textbf{Kerm:} Sure, absolutely. It's in the striving for that goal, in
that confrontation with the malevolent elements within ourselves, that
we might finally find some grace---

\textbf{Ewald:} Excuse me, I saw that you've been waiting here but
didn't get on the bus. The city passed a law that made loitering around
bus stops illegal. If you're not waiting for a bus, I'm going to have to
ask you all to leave.

\textbf{Lucian:} I know what you are, all of you, from first to
last---you are the people in power! You are the police---the great fat,
smiling men in blue and buttons! You are the Law, and you have never
been broken. But is there a free soul alive that does not long to break
you, only because you have never been broken? You sit in your empty
chairs, and have never come down from them. You have had no troubles.
Oh, I could forgive you everything, you that rule all mankind, if I
could feel for once that you had suffer---
\autocite[326-7]{chesterton_thursday}

\begin{center}\rule{0.5\linewidth}{0.5pt}\end{center}

The first side of the tape ends abruptly here. The other side of the
tape was blank.
