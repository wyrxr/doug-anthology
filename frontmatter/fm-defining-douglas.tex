Douglas, colloquially known as DougDoug, is the central signifier within
the broader context of Douglas Studies. Importantly, Douglas and Doug
serve as specific terms of art whose definitions are less important than
their value for distinguishing between respected academics and the
uneducated rube. Before committing to a concrete definition, we must
first explore the primary schools of thought underpinning the various
definitions of ``Douglas.''

\subsubsection{THE ESSENTIALIST SCHOOL}\label{the-essentialist-school}

The Essentialist School of Douglas
Studies\index{Essentialist School of Douglas Studies} posits that there
are various inherent ``Dougnesses'' which define an object, such as
``bald'' and ``bad at 2D platformers.''\index{Bad at 2D Platformers}
These ``Dougnesses'' collectively place the object within a Great Chain
of Doug based on their Douglikeness. Doug Essentialists view everything
as containing some degree of Douglikeness, even if the object has almost
no propinquity to the ideal Doug.

The major criticism this school receives is its tautological nature. We
know a Doug because it has certain attributes, and we know that those
attributes are the right ones because a Doug has them. This kind of
question-begging finds strong critiques from scholars who want to
question some or all of the assumed values these ``Dougnesses'' embody.

\subsubsection{THE HISTORICIST SCHOOL}\label{the-historicist-school}

The Historicist School\index{Essentialist School of Douglas Studies}
views Douglas as inseparable from the historical context that produced
it---specifically, the real, 21st-century Twitch streamer and YouTuber
Douglas ``DougDoug'' Scott Wreden and his community. These
physical-historical facts take precedence over artistic, philosophical,
or metaphysical concerns. These scholars often fight back against what
they see as ``encroachment,'' or the development of the field beyond its
historical boundaries. They are primarily concerned with identifying and
eliminating ``accretions,'' i.e., the supposedly inauthentic traditions
or interpretations of later scholars.

The major criticism against this school often centers around its
inability to adapt to evolving knowledge in the field. As the famed
Douglasian scholar Grimden Penrose puts it: ``They treat us like a
marble statue. Everything's already there, they just gotta chip away at
the `pseudo-intellectual' exterior. But we are not {[}\ldots{]} marble.
We are a shrub trying to grow into a mighty tree, though struggling
under {[}their{]} constipant {[}sic{]} haranguing and pruning''
\autocite[42]{penrose_shrub}.

\subsubsection{THE EVOLUTIONARY PSYCHOLOGY
SCHOOL}\label{the-evolutionary-psychology-school}

Unsurprisingly, the Evolutionary Psychology
school\index{Evolutionary Psychology School of Douglas Studies} views
``Douglas'' as a necessary component of our survival. Ancient humans
encountered a world where ``Douglas'' proved useful, and thus those
individuals which expressed ``Douglas'' were more fit than their
non-Doug counterparts. The proponents of this school also view ``Doug''
as a psychological phenomenon that people experience but which is not
real in a metaphysical sense. These experiences are actually derived
from hallucination, mass psychosis, or trickery, each of which served an
important role in the ancestral environment. Several scholars from this
school have protested against public funding for Doug-based education.

The deceptive simplicity of this school makes it especially popular in
the modern discourse. However, critics have identified some important
caveats. This school views the existence of ``Douglas'' in the present
as proof that it must have survived the selective pressures of the past,
but there is no historical evidence for ``Douglas'' in prehistoric man
or at any point which could be influential on an evolutionary time
scale.

\subsubsection{THE SOCIAL CONSTRUCTIVIST
SCHOOL}\label{the-social-constructivist-school}

The Social
Constructivist\index{Social Constructivist School of Douglas Studies}
school views ``Douglas'' as a social construct. Communities, such as a
fan club, academic faculty, or Congress, collaboratively define concepts
and impose their definition through usage and communicative policing.
Rival definitions are made illegible through the violence of
misunderstanding. Systems of power load up their lexical cannons and
performatively blast at the semantic bastions of their socio-economic
enemies. When the walls of meaning crumble, they rush inside and
indiscriminately slaughter the innocent neologisms, idiolects, and
burgeoning slang words sleeping in their cradles. The fires spread, the
destruction mounts, the days are long, and the few words left can't
suffice to account for what was done.

This school no longer receives substantial criticism, which
substantially reduced the number of fist fights on campus.

\subsubsection{REVIEW}\label{review}

As you can see, ``Douglas'\,' means many things to many people. As you
read the following essays, keep this always in mind: ``What does Doug
mean to \emph{you}?''
