The rapidity with which Douglas Studies has evolved over the years since
the publication of the Second Edition has confounded both its
practitioners and critics. The field has changed so much that the name
Douglas Studies itself seems like a misnomer. We are living in a
Post-Douglas Studies world. Douglasian theories provide the
philosophical operating system for modern society. Those outside its
embrace must define themselves through opposition or absence, an
apophatic theology derived from the very Doug they deny.

This Third Edition represents a radical departure from previous
editions. Despite how beloved and culturally important the material has
become, the editors must confront how deeply outdated and problematic
these texts truly are. The hegemonic assumptions that produced this
selection cannot be uncritically perpetuated, not least of all in a
serious academic anthology. This has led to a complete rethinking and
redesigning of both the inclusion criteria and the panel of judges and
editors assigned to produce the final selection.

The present edition is particularly sensitive to ensuring perspectives
that were historically marginalized (the Strikers, Lawyerites, and VOD
Chat among others). While a few works from the previous editions deemed
``absolutely essential'' have been retained, the remainder represents
the current vanguard of Douglas scholarship. While we do not expect the
student to agree with every essay presented, we hope that by
encountering these diverse perspectives a perceptive reader will be able
to question their own assumptions and participate in the productive
contradictions that modern discourse demands.
